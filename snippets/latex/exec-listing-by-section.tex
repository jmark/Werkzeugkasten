\documentclass[11pt,a4paper]{book}

\usepackage{amsmath}
\usepackage[english]{babel}
\usepackage{blindtext}

% =========================================== %
% =========================================== %
% Das muss in die Präambel.

\newcounter{exercisectr}
\newenvironment{exercise}{
    \bigskip
    \footnotesize
    \noindent
    \refstepcounter{exercisectr}
    \textbf{Exercise \theexercisectr}
}{}
\numberwithin{exercisectr}{chapter}

% =========================================== %
% =========================================== %

\begin{document}
\tableofcontents

% =========================================== %

\chapter{Title of Chapter A}
\blindtext 

For a deeper understanding we refer to exercise \ref{exec:1a} in this section
and exercise \ref{exec:2b} in the following one.

\section{Section A}
\blindtext

% =========================================== %
% So wird eine Übung erstellt.

\begin{exercise}
\label{exec:1a}
    \blindtext
\end{exercise}

\begin{exercise}
\label{exec:2a}
    \blindtext
\end{exercise}

\begin{exercise}
\label{exec:3a}
    \blindtext
\end{exercise}

\chapter{Title of Chapter B}

\section{Section B.a}
\blindtext

\section{Section B.b}
\blindtext

\begin{exercise}
\label{exec:1b}
    \blindtext
\end{exercise}

\begin{exercise}
\label{exec:2b}
    \blindtext
\end{exercise}

\end{document}
