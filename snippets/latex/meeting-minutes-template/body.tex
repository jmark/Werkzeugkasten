\topBox{
\item[Zeit/Ort:] 
24.01.2012 - 10:30 Uhr bis 11:25 Uhr, P031 / Physikinstitut

\item[Anwesende Mitglieder: (10/12)]
Lars Bonitz, Martin Dehnert, Katrin Bockhoff, \\
Johannes Meiwald, Benno Ohme, Martin Hartmann, Susanne Müller,\\
Phillip Böttger, Hans Weber, Johannes Markert 

\item[Entschuldigt: (2)]
Winfried Leistner, Julia Seemann

\item[Gäste: (4)] 
Arnt Roscher, Constanze Greger, Jennifer Brade, Karoline Griesbach
}

\customTop{Obligatorisches}{
\item
Feststellen der Beschlussfähigkeit
\item
Letztes Sitzungsprotokoll (17.01.2012) wurde angenommen. \Abstimmung{10}{0}{0}
}

\customTop{Sturawahlen}{
\item 
Martin D., als Sturamitglied, berichtet: Alles verläuft planmäßig. 
Nur der neue FSR, welcher erst im April seine Arbeit aufnehmen wird,
ist für den neuen Stura wahlberechtigt.
\item
Jeder der Lust hat, kann sich Sturawahl aufstellen lassen. (max. 2 aus der 
Fachschaft Physik dürfen letztendlich in den Stura).
\item
Martin Dehnert, Jennifer Brade und Karoline Griesbach stellen 
sich und ihre Motivation, Sturamitglieder zu werden, vor.
}

\customTop{Faschings-Angrillen}{
\item Martin H. erläutert die Kalkulation.
\item Die Kalkulation zur Faschingsfeier am 26.01.2012 wurde angenommen.
        \Abstimmung{10}{0}{0}
\item Beginn: Aufbau 17:00 Uhr 
\item Konkrete Aufgabenverteilung dann am Abend.
}

\vspace{0.6cm}
\emph{11:56 Uhr \ \  
Martin Dehnert und Johannes Meiwald verlassen die Sitzung.
}

\customTop{Tischkicker}{
\item Bisher keine neuen Entwicklungen.
\item Idee: Bälle gegen Pfand verleihen.
\item allgemein: Geld darf nicht eingenommen werden.
}

\customTop{Volleyballturnier}{
\item
Zitat aus der Email von Johannes Meiwald 24.01.2012:
"`Der FSR Physik Dresden veranstaltet am 14.04.2012 (Samstag) ein 
Volleyball-Tournier und wir haben einen Startplatz. Deswegen möchte ich gern 
wissen ob Interesse besteht dort mit zu machen (Team stellen)."'
\item
Wann: Samstag, 14.04.2012, 
Wo? in Dresden
\item
Es besteht weitreichendes Interesse. Die Chancen ein motiviertes Team
aufzustellen stehen gut.
\item
Ansprechpartner: Johannes Meiwald
}

\customTop{Sonstiges}{
\item 
Mit dem Benehmen des FSR Physik wird Sarah Schubert Mitglied der Stuko 
Sensorik und kognitive Psychologie.
\item  
Nächste außerplanmäßige Sitzung: Mittwoch, 01.02.2012, 13:00 Uhr, Seminarräumen \\
Themen:
Konstuierende Sitzung des neuen FSRs vor der
allgemeinen Sitzung, Stura-Wahl, Finanzer;\\
Sitzungsleitung: Martin Dehnert
}

\vspace{2cm}

\Unterschrift{Lars Bonitz}{Johannes Markert}

